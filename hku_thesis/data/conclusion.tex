\iffalse
\bibliography{myreference.bib}
\fi
%chapter 6

\chapter{Conclusion}
\section{Conclusions}
In this project, we use linear model and tree model to indicate the importance of features. We use CFS to remove the bias from correlation between input features. We also use permutation-based random forest to prevent the bias from different number of category. We also have evaluate the performance of model we train. The good performance also support the correctness of feature importance.

We find that linear model is convenient and feasible to represent the importance of feature by their coefficient. But it is prone to suffer the bias on multiple correlated feature regarding their importance. Thus if we want to use linear model, we need to determine that there is little correlation between input features.

We also find that random forest can reduce the bias of correlation between input features. But it also has a bias on the different level of category on features. We find that permutation-based random forest can work well on this situation and we find that the feature ``BlueV'' is important using this method.

\section{Future Work}
We think correlation between feature still have bias. We can find more methods to remove it.

We can also extract more feature to analyze using natural language processing technique. 

Concerning the dataset, we can also use some outlier detection technique to remove noise or outlier that make our evaluation more accurate.

Actually this project is only related to theoretical research. In the future, we can use the knowledge found in this project to build an application and help us identify the credibility of Weibo in our daily life.