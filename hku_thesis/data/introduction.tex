\iffalse
\bibliography{myreference.bib}
\fi
% introduction

\chapter{Introduction}
\section{Motivation}
Micro-blogging has become a main approach for people to acquire information outside. In US, people mainly use Twitter as the social tool. But Sina Weibo is most popular social network used in China, with over 500 million registered users, which allows users to broadcast up to 140 characters message with multimedias like images and videos. Those messages can be reposted or commented by the followers of the authors. However, a large amount of messages contain fake information. Once these messages are diffused fast and immoderately, they will cause a huge damage on the stability of the society. These messages are called romors, which need to satisfy two necessary requirement. First, it must be a false message. Second, its repost number should be larger than a specific value. Otherwise, even it is a false message, it can not be concerned as a rumor if nobody receives it. 

Due to the property of fast transmission, real-time and no reviewing, Sina Weibo provides the convenience for the creation and transmission of rumor. Whether there is a breaking-news or not, Sina Weibo is filled with various rumors. According to the statistic in this Blue Book\cite{Chinareport}, among the 100 hot news, one third of Weibo about them are rumors. March 2012, the rumor stating ``Military convoys get into Beijing and something happens'' made 16 related websites shutdown. Sina Weibo had to close the comment function\footnote{\url{http://mil.huanqiu.com/Observation/2012-04/2580647.html}}. Because of this and other similar rumor incidents, building a rumor automatic detection system on social network is necessary and significant.
\section{Previous Work}
Some researchers have researched on Twitter. Castillo et al.\cite{castillo2011information} focus on automatic methods for assessing the credibulity of a given set of tweets. They analyze tweets related to ``trending'' topics and classify them as credible or not credible. Their method is based on supervised learning including SVM, J48 decision tree and Bayes networks. They also do the best-first feature selection and search forward. Qazvinian et al.\cite{qazvinian2011rumor} explore the effectiveness of 3 categories of features including content-based, network-based and microblog-specific memes for correctly identifying rumors. They also show how these features are effective in identifying users who endorse a rumor and further help it to spread. They build different Bayes classifiers as high level features and learning a linear function of these classifiers for retrieval in these two parts.

Recently, there are also some research related to Sina Weibo. Yang et al. collect an extensive set of microblogs from official rumor-busting service provided by Sina Weibo and extract an extensive set of features for training a classifier to automatically detect the rumors from a mixed set of true information and false information. This is the first stydy on rumor analysis and detection on Sina Weibo\cite{yang2012automatic}. Wu et al. stand on a total different viewpoint to study the problem of automatic detection of false rumors on Sina Weibo\cite{wu2015false}. Not like the traditional feature-based approaches extracting features from the false rumor message such as its author as well as the statistics of its responses to form a flat feature vector, they propose a graph-kernel based hybrid SVM classifier for analyzing the propagation structure of the messages. They also extract semantic feature such as topics and sentiments.

\section{Contributions}
Since we want to build a automatical rumor detection system, we need to crawl rumor dataset and extract features from them. Using these features, a rumor detection model can be trained. However, there are so many features, what feature can best describe the structure inherent in the data? In other words, since we are facing the problem of rumor detection, which feature is most helpful or has the most contribution for us to do accurate classification?

The following are what we have done in this project:\begin{enumerate} \item We survey the research work related to the rumor detection on Twitter and Sina Weibo, which has been decribed above.\item We also survey the method to analyze features.\item We crawl dataset from Sina Weibo. \item We extract features and train two models. With the help of models, we figure out the importance of features.\item We evaluate the performance of models we train.
\end{enumerate}
\section{Report Organization}
The rest of this report will be organized using 6 chapters. We first introduce what is feature analysis and its benefit in chapter 2. In the chapter 3, we will discuss the procedure of how we build the dataset. In the chapter 4, the principal theory of algorithm we use will be introduced. In the chapter 5, we will discuss the detailed experiment procedure and we draw conclusions and propose future work in chapter 6.
